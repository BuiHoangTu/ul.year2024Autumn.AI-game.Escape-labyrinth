\documentclass[a4paper,12pt]{article}

% Packages
\usepackage[utf8]{inputenc}
\usepackage{geometry}
\usepackage{graphicx}
\usepackage{hyperref}
\usepackage{amsmath}
\usepackage{listings}
\usepackage{xcolor}

% Page Setup
\geometry{margin=1in}

% Code Snippet Settings
\lstset{
    basicstyle=\ttfamily\small,
    backgroundcolor=\color{lightgray!20},
    frame=single,
    breaklines=true,
    postbreak=\mbox{\textcolor{red}{$\hookrightarrow$}\space},
    numbers=left,
    numberstyle=\tiny\color{gray},
    keywordstyle=\color{blue},
    commentstyle=\color{green!70!black},
    stringstyle=\color{red},
}

% Title Section
\title{Unity RTS Unit Report: Pathfinding and FSM \\ \small Word Count: XXX \\ Student ID: XXXXXXXX}
\author{Your Name}
\date{\today}

\begin{document}

% Title Page
\maketitle
\tableofcontents
\newpage

% Sections
\section{Introduction}
\subsection{Game Overview}
Introduce the RTS unit as the artefact, detailing the AI techniques implemented (Pathfinding and FSM) to solve the problem of unit behavior in an RTS game.

\subsection{Comparative Analysis}
Compare your implementation with similar mechanics in contemporary games and insights from literature. Mention any inspiration or benchmarks you aimed to meet.

\section{Analysis}
\subsection{Rationale for Techniques}
Explain why you chose specific techniques (e.g., NavMesh for pathfinding, FSM for behavior modeling). Back up your choices with references to academic research or practical considerations.

\subsection{Understanding and Implementation}
Demonstrate your understanding of the chosen techniques by providing:
\begin{itemize}
    \item A brief explanation of the concepts (e.g., how FSMs work).
    \item High-level details of how you implemented these techniques in Unity.
\end{itemize}

\section{Reflection}
\subsection{Results and Behavior}
Discuss the outcomes of your implementation:
\begin{itemize}
    \item How well the RTS unit behaved according to expectations.
    \item Examples of scenarios where the AI worked as intended.
\end{itemize}

\subsection{Issues and Limitations}
Highlight challenges faced during implementation, such as:
\begin{itemize}
    \item Edge cases (e.g., units getting stuck during pathfinding).
    \item Limitations of the FSM model or the pathfinding system.
\end{itemize}

\subsection{Alternative Solutions and Improvements}
Suggest possible improvements or alternative techniques, referencing research or practical solutions. Discuss future directions for enhancing unit AI.

\section{Conclusion}
\subsection{Summary}
Summarize the contributions of the report, including insights into the techniques and their implementation.

\subsection{Critical Reflection}
Reflect on your individual performance in the project:
\begin{itemize}
    \item What went well.
    \item Areas for improvement.
\end{itemize}

\section*{References}
List all references used in the report in a standard format (e.g., IEEE, APA).

\section*{Appendices (Optional)}
\subsection*{Code Snippets}
Include any relevant code snippets or detailed diagrams.

\end{document}
